\documentclass[12pt, spanish, oneside, onecolumn, a4paper]{report}
\usepackage[spanish,activeacute]{babel}
\usepackage[latin1,utf8]{inputenc}
\usepackage{times}
\usepackage[T1]{fontenc}
\usefont{T1}{arial}{m}{n}
\oddsidemargin 0in
\textwidth 6.75in
\topmargin 0in
\textheight 8.5in
% \parindent 0em
\parskip 2ex

\usepackage{fancyheadings}
\headheight 35pt
\usepackage{graphicx}
\usepackage[colorlinks=true, urlcolor=blue, filecolor=green,
linkcolor=red, pdfkeywords={}, pagebackref, pdfpagemode=UseOutlines,
bookmarksopen=true]{hyperref}

\usepackage{tabularx}
\usepackage{colortbl}

\usepackage{tabulary}
\setlength\tymin{699pt}
\setlength\tymax{700pt}
\setlength\doublerulesep{0.5px}

\graphicspath{{./img/}}

\usepackage{epic}
\makeatletter

\renewcommand{\contentsname}{Indice}
\renewcommand{\appendixname}{Apéndice}
\renewcommand{\figurename}{Figura}
\renewcommand{\listfigurename}{Indice de figuras}
\renewcommand{\tablename}{Tabla}
\renewcommand{\listtablename}{Indice de tablas}

% un comando
\def\comando#1{

\begin{tabulary}{620px}{|>{\columncolor[rgb]{0.8,0.9,0.8}}L|}
  \hline \textsl{#1}\ \hline
\end{tabulary}

}

% varios comandos
\newenvironment{command} {

  \tabulary{420px}{*{2}{|>{\columncolor[rgb]{0.8,0.9,0.8}}L}|}
  % \hline
  \textbf{portage:} & \textbf{entropy:}\tabularnewline } {
  % \hline
  \endtabulary

}
% end comandos

\usepackage[toc]{glossaries}
\makeglossaries

%\begin{titlepage}
%\newcommand{\HRule}{\rule{\linewidth}{0.5mm}} % Defines a new command for the horizontal lines, change thickness here
%\center % Center everything on the page
%
%%\textsc{\LARGE University Name}\\[1.5cm] % Name of your university/college
%%\textsc{\Large Major Heading}\\[0.5cm] % Major heading such as course name
%%\textsc{\large Minor Heading}\\[0.5cm] % Minor heading such as course title
%\HRule \\[0.4cm]
%{ \huge \bfseries Git}\\[0.4cm] % Title of your document
%\HRule \\[1.5cm]
%\begin{minipage}{0.4\textwidth}
%\begin{flushleft} \large
%\emph{Author:}\\
%Ing. Anielkis \textsc{Herrera} % Your name
%\end{flushleft}
%\end{minipage}
%~
%\begin{minipage}{0.4\textwidth}
%\begin{flushright} \large
%%\emph{Supervisor:} \\
%%Dr. James \textsc{Smith} % Supervisor's Name
%\end{flushright}
%\end{minipage}\\[4cm]
%{\large \today}\\[3cm] % Date, change the \today to a set date if you want to be precise
%%\includegraphics{Logo}\\[1cm] % Include a department/university logo - this will require the graphicx package
%\vfill % Fill the rest of the page with whitespace
%\end{titlepage}

\makeatletter
\def\maketitle{%
  \null
  \thispagestyle{empty}%
  \vfill
  \begin{center}\leavevmode
    \normalfont
    \includegraphics{logo.png}\\[1cm]
%    {\LARGE \@title\par}%
    \vskip 1cm
    {\Large \@author\par}%
    \vskip 1cm
    {\Large \@date\par}%
  \end{center}%
  \vfill
  \null
  \cleardoublepage
  }
\makeatother


\begin{document}
\pagestyle{fancy}
\lhead{\includegraphics[width=3cm,keepaspectratio=true]{logo.png}
%\includegraphics[width=3cm,keepaspectratio=true]{emblema-2008.eps}
}
%\cfoot{ \itshape{Carretera a San Antonio de los Baños. Km 5 {1/2}
% Reparto Torrens. Boyeros. Ciudad Habana.  Teléfono (53 7) 8372519
% e-mail: decano.f10@uci.cu }\newline \thepage}

\title{ Git }
\date{ \today\ }
\author{ Ing. Anielkis Herrera }

\pagenumbering{roman}
\maketitle

%\begin{abstract}
%Guía de uso de la herramienta Git para control de versiones
%\end{abstract}

\newpage
\pagenumbering{arabic}


\chapter{Introducción a Git}
\label{chap:intro}


\newacronym[longplural=Sistemas de Control de Versiones]{scv}{SCV}{Sistema de Control de Versiones}

%\newglossaryentry{scv}
%{
%  description={Sistema de Control de Versiones, sistemas utilizados para gestionar versionado en el desarrollo de software o documentación}
%}

%\newglossaryentry{Linux}
%{
%  description={es un término genérico que se refiere a la familia de sistemas operativos ``parecidos a Unix'' que utilizan el núcleo Linux}
%}

Git es un sistema de control de versiones distribuído, desarrollado como sistema de software libre y diseñado para manejar desde proyectos pequeños hasta muy grandes, con gran rapidez y eficiencia.

Git es fácil de aprender y tiene una huella pequeña en el sistema, con un rendimiento increíblemente rápido. Supera a otros \glspl{scv} como Subversion , CVS , Perforce y ClearCase con características como trabajo con ramas locales, áreas de ``puesta en escena'' muy convenientes, y varios estilos de flujo de trabajo.


\section{Acerca de Git}
\label{sec:aboutgit}

\subsection{Creación y fusión de ramas}
\label{sec:branchingandmerging}

La función de Git que realmente lo hace destacar de entre la mayoría de los \gls{scv} es su modelo de ramificación. Git permite y alienta a tener varias ramas locales que pueden ser completamente independientes una de otra. La creación, fusión y supresión de las líneas de desarrollo toma solo unos instantes de segundo.

Esto significa que se pueden hacer cosas como:
\begin{description}
\item [Cambio de contexto sin fricción:] crear una rama para probar una idea,
adicionar contenido al historial un par de veces, volver a la rama desde donde se ramificó, aplicar un
parche, cambiar de nuevo a donde se está experimentando, y fusionar las ramas para adicionar los cambios del experimento a la rama original.
\item [Líneas de desarrollo separadas en roles:] tener una rama que siempre contiene sólo lo que va a la producción, otra a dónde se fusionan el código en que se ha trabajado, para probarlo, y varios más pequeños para el día a día.
\item [Función de flujo de base:] crear nuevas ramas para cada nueva característica en que se esté trabajando por lo que perfectamente puede alternar entre ellas, y a continuación eliminar cada rama cuando esta característica se fusione en la línea principal.
\item [Experimentación desechable:] crear una rama en la que experimentar , darse cuenta de que no va funcionar el experimento, y simplemente eliminarla - el abandono de un trabajo fallido que nadie más va a ver.
\end{description}

\includegraphics[width=5cm,keepaspectratio=true]{branches.png}


Notably, when you push to a remote repository, you do not have to push all of your branches. You can choose to share just one of your branches, a few of them, or all of them. This tends to free people to try new ideas without worrying about having to plan how and when they are going to merge it in or share it with others.

There are ways to accomplish some of this with other systems, but the work involved is much more difficult and error-prone. Git makes this process incredibly easy and it changes the way most developers work when they learn it. 


\glsaddall
\printglossaries

\end{document}
